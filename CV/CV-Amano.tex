\documentclass[lualatex,a4paper,ja=standard]{scrartcl}
\usepackage[english]{babel}
\usepackage[protrusion=true,expansion=true]{microtype}
\usepackage{amsmath,amsfonts,amsthm}     % Math packages
\usepackage{graphicx,color}    % Enable pdflatex
\usepackage[svgnames]{xcolor}            % Colors by their 'svgnames'
\usepackage{geometry}
%\textheight=700px                    % Saving trees ;-)
\usepackage{url}
\usepackage[colorlinks=true,linkcolor=blue,urlcolor=blue]{hyperref}
\usepackage{float}
\usepackage{etaremune}
\usepackage{wrapfig}
\usepackage{natbib}
\usepackage{bibentry}
\usepackage{xstring}
\usepackage{seqsplit}
\usepackage[en-US]{datetime2}
\usepackage{attachfile}

\frenchspacing              % Better looking spacings after periods
\pagestyle{empty}           % No pagenumbers/headers/footers

%\addtolength{\voffset}{-40pt}
%\addtolength{\textheight}{20pt}

\setlength\topmargin{0pt}
\addtolength\topmargin{-\headheight}
\addtolength\topmargin{-\headsep}
\setlength\oddsidemargin{0pt}
\setlength\textwidth{\paperwidth}
\addtolength\textwidth{-2in}
\setlength\textheight{\paperheight}
%\addtolength\textheight{-3in}
\addtolength\textheight{-2in}
\usepackage{layout}

%%% Custom sectioning}{sectsty package)
%%% ------------------------------------------------------------
\usepackage{sectsty}

\sectionfont{%			            % Change font of \section command
	%\usefont{OT1}{phv}{b}{n}%		% bch-b-n: CharterBT-Bold font
	\sectionrule{0pt}{0pt}{-5pt}{3pt}}

%%% Macros
%%% ------------------------------------------------------------
\newcommand{\sepspace}{\vspace*{1em}}		% Vertical space macro

\newcommand{\MyName}[1]{
		\Huge \hfill \textbf{#1}
        \vspace*{0.5em}
		\par \normalsize \normalfont}

\newcommand{\MySlogan}[1]{
		\large #1
        \vspace*{0.5em}
		\par \normalsize \normalfont}

\newcommand{\NewPart}[2]{\section*{\uppercase{#1} #2}}

\newcommand{\NewSubPart}[1]{\subsection*{#1}}

\newcommand{\GeneralEntry}[4]{
\noindent\hangindent=2em\hangafter=0
\parbox{#1}{#3}
\hspace{1em}
\parbox{#2}{#4} \par
\vspace{0.5em}
}

\newcommand{\TalkEntry}[3]{
\item #1, \textit{#2}, #3.}

%%%
%%% for bibliography
%%%
\makeatletter
\renewcommand\BR@b@bibitem[2][]{\BR@bibitem[#1]{#2}\BR@c@bibitem{#2}}
\DeclareOldFontCommand{\rm}{\normalfont\rmfamily}{\mathrm}
\DeclareOldFontCommand{\sf}{\normalfont\sffamily}{\mathsf}
\DeclareOldFontCommand{\tt}{\normalfont\ttfamily}{\mathtt}
\DeclareOldFontCommand{\bf}{\normalfont\bfseries}{\mathbf}
\DeclareOldFontCommand{\it}{\normalfont\itshape}{\mathit}
\DeclareOldFontCommand{\sl}{\normalfont\slshape}{\@nomath\sl}
\DeclareOldFontCommand{\sc}{\normalfont\scshape}{\@nomath\sc}
\makeatother
\def\FormatName#1{%
  \IfSubStr{#1}{Amano}{\textbf{\underline{#1}}}{#1}%
}
\newcommand{\self}[1]{\textbf{\underline{#1}}}

%%%
%%% font configuration
%%%
\usepackage{fontspec}
\usepackage{luatexja-fontspec}
\setmainfont[BoldFont=NotoSerif-Bold,AutoFakeSlant=0.25]{NotoSerif-Medium}
\setsansfont[BoldFont=NotoSans-Bold,AutoFakeSlant=0.25]{NotoSans-Medium}
\setmainjfont[BoldFont=HaranoAjiMincho-Bold.otf,AutoFakeSlant=0.25]{HaranoAjiMincho-Medium.otf}
\setsansjfont[BoldFont=HaranoAjiGothic-Bold.otf,AutoFakeSlant=0.25]{HaranoAjiGothic-Medium.otf}

%%%
%%% Begin Document
%%%
\begin{document}

%\layout

% you can upload a photo and include it here...
%\begin{wrapfigure}{l}{0.5\textwidth}
	%\vspace*{-2em}
	%	\includegraphics[width=0.2\textwidth]{Appelbaum.JPG}
%\end{wrapfigure}

\renewcommand\labelenumi{[\theenumi]}
\renewcommand\thesubsection{\arabic{subsection}}

%%% WOS citation
\newif\ifWOS
%\WOStrue
\WOSfalse
\ifWOS
\newcommand{\strong}[1]{\textbf{#1}}
\newcommand{\numcitation}[1]{\strong{(Citation: #1)}}
\else
\newcommand{\numcitation}[1]{}
\fi

%%% private info
\newif\ifprivate
%privatetrue
\privatefalse

\def\myphone{+81--3--5841--1921}
\def\myfax{+81--3--5841--8321}
\def\myemail{amano@eps.s.u-tokyo.ac.jp}
\def\mywebsite{https://amanotk.github.io/}

\MyName{
\begin{flushright}
Takanobu Amano\\
%天野 孝伸 (あまの たかのぶ)
\end{flushright}}
\MySlogan{
\begin{flushright}
Associate Professor\\
Department of Earth and Planetary Science,\\
Graduate School of Science, The University of Tokyo\\
\end{flushright}
}

%%% Personal Details
\NewPart{Personal Details}{}

\ifprivate
\GeneralEntry{8em}{32em}{Date of Birth}{January 24, 1981}

\GeneralEntry{8em}{32em}{Gender}{Male}
\fi

\GeneralEntry{8em}{32em}{Affiliation}
{Department of Earth and Planetary Science,\\
Graduate School of Science, The University of Tokyo}

\GeneralEntry{8em}{32em}{Address}
{Room 803, Faculty of Science Bldg. 1,\\
7-3-1 Hongo, Bunkyo-ku, Tokyo, 113-0033, JAPAN.}

\GeneralEntry{8em}{32em}{E-mail}{\myemail}

\GeneralEntry{8em}{32em}{Phone}{\myphone}

\GeneralEntry{8em}{32em}{Fax}{\myfax}

\GeneralEntry{8em}{32em}{Website}{\url{\mywebsite}}

%%% Research Interests
\NewPart{Research Interests}{}
Takanobu Amano is interested in theoretical aspects of space and astrophysical plasma phenomena. His major research interests include physics of collisionless shocks (both non-relativistic and relativistic regimes), high-energy particle acceleration and transport, linear and nonlinear theory for kinetic plasma instabilities, and numerical techniques for advanced kinetic/fluid plasma simulations.

\clearpage

%%% Appointments
\NewPart{Appointments}{}

\GeneralEntry{14em}{28em}{Aug. 1, 2016 - present}
{\textbf{Associate Professor}\\
Department of Earth and Planetary Science,\\
School of Science, The University of Tokyo}

\GeneralEntry{14em}{28em}{Mar. 16, 2012 - Jul. 31, 2016}
{\textbf{Assistant Professor}\\
Department of Earth and Planetary Science,\\
School of Science, The University of Tokyo}

\GeneralEntry{14em}{28em}{Apr. 1, 2009 - Mar. 15, 2012}
{\textbf{Designated Assistant Professor}\\
Division of Particle and Astrophysical Science,\\
Nagoya University}

\GeneralEntry{14em}{28em}{Apr. 1, 2008 - Mar. 31, 2009}
{\textbf{Postdoctoral Researcher}\\
Solar-Terrestrial Environment Laboratory,\\
Nagoya University}


%%% Education
\NewPart{Education}{}

\GeneralEntry{14em}{28em}{Apr. 1, 2005 - Mar. 31, 2008}
{\textbf{Ph.D degree}\\
Department of Earth and Planetary Science,\\
Graduate School of Science, The University of Tokyo}

\GeneralEntry{14em}{28em}{Apr. 1, 2003 - Mar. 31, 2005}
{\textbf{MS degree}\\
Department of Earth and Planetary Science,\\
Graduate School of Science, The University of Tokyo}


\GeneralEntry{14em}{28em}{Apr. 1, 1999 - Mar. 31, 2003}
{\textbf{BS degree}\\
Department of Earth and Planetary Physics,\\
School of Science, The University of Tokyo}


%%% Awards
\NewPart{Awards}{}
\begin{itemize}
\item 2022 Tanakadate Award from from Society of Geomagnetism and Earth, Planetary and Space Sciences (SGEPSS)
\item 2018 Young Researcher Award (under 40 yrs. old) from Association of Asia Pacific Physical Societies, Division of Plasma Physics (AAPPS-DPP)
\item 2015 Obayashi Early Career Scientist Award from Society of Geomagnetism and Earth, Planetary and Space Sciences (SGEPSS)
\item 2005 JSPS (Japan Society for the Promotion of Science) Research Fellowship for Young Scientists (DC1)
\end{itemize}


\clearpage
%%% Papers
\NewPart{Publications}{}

See also,
\href{https://scholar.google.co.jp/citations?hl=ja&user=o23rFB8AAAAJ&view_op=list_works&sortby=pubdate}{Google Scholar}, or
\href{https://publons.com/researcher/2523588/takanobu-amano/}{Publons} profile pages
for the up-to-date list of publications and citation statistics.

\ifWOS
% summary
\NewSubPart{Summary of Publications}

\GeneralEntry{28em}{8em}{Number of Refereed Publications (First Author)}{13}
\GeneralEntry{28em}{8em}{Number of Refereed Publications (Total)}{34}
\GeneralEntry{28em}{8em}{Total Number of Citations}{295}
\GeneralEntry{28em}{8em}{H-Index}{10}

The citation statistics (as of \today) was taken from Web of Science.
\fi

% list of peer-reviewed
%\bibliographystyle{agu-custom}
\bibliographystyle{chicago-custom}
\nobibliography{publication}

\NewSubPart{Refereed Articles}
\begin{enumerate}
% 2024
\item \bibentry{Walia2024a} \numcitation{0}
\item \bibentry{Kataoka2024a} \numcitation{0}
\item \bibentry{Amano2024a} \numcitation{0}
\item \bibentry{Jikei2024b} \numcitation{0}
\item \bibentry{Lindberg2024a} \numcitation{0}
\item \bibentry{Boula2024a} \numcitation{0}
\item \bibentry{Jikei2024a} \numcitation{0}
\item \bibentry{Iwamoto2024a} \numcitation{0}
% 2023
\item \bibentry{Yamakawa2023a} \numcitation{0}
\item \bibentry{Raymond2023a} \numcitation{0}
\item \bibentry{Kuramitsu2023a} \numcitation{0}
% 2022
\item \bibentry{Kitamura2022a} \numcitation{0}
\item \bibentry{Amano2022b} \numcitation{0}
\item \bibentry{Yamakawa2022a} \numcitation{0}
\item \bibentry{Walia2022a} \numcitation{0}
\item \bibentry{Jikei2022a} \numcitation{0}
\item \bibentry{Amano2022a} \numcitation{0}
\item \bibentry{Iwamoto2022a} \numcitation{0}
\item \bibentry{Keika2022a} \numcitation{0}
% 2021
\item \bibentry{Kobzar2021a} \numcitation{0}
\item \bibentry{Nishigai2021a} \textbf{(Corresponding Author)} \numcitation{0}
\item \bibentry{Kitamura2021a} \numcitation{0}
\item \bibentry{Jikei2021a} \numcitation{0}
\item \bibentry{Ligorini2021a} \numcitation{0}
\item \bibentry{Bohdan2021a} \numcitation{0}
\item \bibentry{Ligorini2021b} \numcitation{0}
% 2020
\item \bibentry{Bohdan2020b} \numcitation{0}
\item \bibentry{Yamakawa2020a} \numcitation{0}
\item \bibentry{Kitamura2020a} \numcitation{0}
\item \bibentry{Bohdan2020a} \numcitation{0}
\item \bibentry{Amano2020a} \numcitation{0}
% 2019
\item \bibentry{Oka2019a} \numcitation{0}
\item \bibentry{Bohdan2019a} \numcitation{0}
\item \bibentry{Iwamoto2019a} \numcitation{0}
\item \bibentry{Bohdan2019b} \numcitation{0}
\item \bibentry{Katou2019a} \numcitation{0}
\item \bibentry{Yamakawa2019a} \numcitation{0}
\item \bibentry{Amano2019a} \numcitation{0}
% 2018
\item \bibentry{Seki2018a} \numcitation{0}
\item \bibentry{Keika2018a} \numcitation{0}
\item \bibentry{Amano2018a} \numcitation{0}
\item \bibentry{Walia2018a} \numcitation{0}
\item \bibentry{Iwamoto2018a} \numcitation{0}
\item \bibentry{Kamiya2018a} \numcitation{0}
% 2017
\item \bibentry{Matsumoto2017a} \numcitation{0}
\item \bibentry{Oka2017a} \numcitation{0}
\item \bibentry{Iwamoto2017a} \numcitation{0}
% 2016
\item \bibentry{Hirabayashi2016a} \numcitation{0}
\item \bibentry{Amano2016a} \numcitation{0}
\item \bibentry{Balsara2016a} \numcitation{0}
% 2015
\item \bibentry{Amano2015a} \numcitation{0}
\item \bibentry{Matsumoto2015a} \numcitation{0}
\item \bibentry{Minoshima2015a} \numcitation{0}
% 2014
\item \bibentry{Itou2014a} \numcitation{0}
\item \bibentry{Amano2014a} \numcitation{0}
% 2013
\item \bibentry{Matsumoto2013a} \numcitation{0}
\item \bibentry{Saito2013a} \numcitation{0}
\item \bibentry{Amano2013a} \numcitation{0}
\item \bibentry{Minoshima2013a} \numcitation{0}
% 2012
\item \bibentry{Matsumoto2012a} \numcitation{0}
\item \bibentry{Umeda2012a} \numcitation{0}
\item \bibentry{Hayakawa2012a} \numcitation{0}
\item \bibentry{Minoshima2012a} \numcitation{0}
\item \bibentry{Amano2012a} \numcitation{0}
% 2011
\item \bibentry{Amano2011b} \numcitation{0}
\item \bibentry{Minoshima2011a} \numcitation{0}
\item \bibentry{Amano2011a} \numcitation{0}
% 2010
\item \bibentry{Amano2010a} \numcitation{0}
\item \bibentry{Shimada2010a} \numcitation{0}
% 2009
\item \bibentry{Amano2009b} \numcitation{0}
\item \bibentry{Amano2009c} \numcitation{0}
\item \bibentry{Amano2009a} \numcitation{0}
% 2008
% 2007
\item \bibentry{Amano2007a} \numcitation{0}
\end{enumerate}

% others
\NewSubPart{Book Chapters}
\begin{enumerate}
\item \bibentry{Amano2023a}
\end{enumerate}

\NewSubPart{Non-Refereed Articles}
\begin{enumerate}
\item \bibentry{Amano2016b}
\end{enumerate}

\NewSubPart{Non-refereed Articles in Japanese}
\begin{enumerate}
\item 星野真弘, \self{天野孝伸} (2009), 宇宙における衝撃波粒子加速機構の新展開, 日本物理学会誌, 64(6), 421
\item \self{天野孝伸} (2009), 超新星残骸衝撃波における電子注入, 天文月報, 102(1), 9
\end{enumerate}

\newpage
\NewPart{Invited Talks (International Conferences)}{}
\begin{enumerate}

\TalkEntry
{Electron injection via stochastic shock drift acceleration at quasi-perpendicular shocks}
{Synergistic approaches to particle transport in magnetized turbulence: from the laboratory to astrophysics}
{Apr. 16, 2024}

\TalkEntry
{Theory, Simulation, and Observation for Electron Injection at Collisionless Shocks}
{AOGS 19th Annual Meeting}
{Online, Aug. 1, 2022}

\TalkEntry
{Electron injection at shocks: Transition from stochastic shock drift acceleration to diffusive shock acceleration}
{XXVIII Cracow EPIPHANY Conference on Recent Advances in Astroparticle Physics}
{Online, Jan. 12, 2022}

\TalkEntry
{Connecting Injection and Subsequent Acceleration of Nonthermal Electrons at Collisionless Oblique Shocks}
{The 30th International Toki Conference on Plasma and Fusion Research (ITC30)}
{Online, Nov. 16, 2021} \textbf{(Plenary Talk)}

\TalkEntry
{Stochastic Shock Drift Acceleration as the Mechanism for Electron Injection into Diffusive Shock Acceleration at Collisionless Shocks}
{5th Asia-Pacific Conference on Plasma Physics (AAPPS-DPP2021)}
{Online, Sep. 28, 2021}

\TalkEntry
{Particle Acceleration at Collisionless Shocks}
{10th East-Asia Workshop on Laboratory, Space, Astrophysical Plasmas (EASW-10)}
{Online, Aug. 16, 2021}

\TalkEntry
{Perspectives for Electron Heating and Acceleration at Collisionless Shocks}
{MMS Spring 2021 Science Working Team Meeting}
{Online, Apr. 8, 2021}

\TalkEntry
{Non-thermal Particle Acceleration at Collisionless Shocks}
{Max Planck Princeton Center Workshop}
{G\"ottingen, Germany, Jan. 22, 2020}

\TalkEntry
{Three-dimensional Particle-In-Cell Simulations for High Mach Number Collisionless Shocks}
{The 2nd Asia-Pacific Conference on Plasma Physics}
{Kanazawa, Japan, Nov. 15, 2018}

\TalkEntry
{Nonthermal Electron Acceleration at Earth's Bow Shock: Theory, Simulation and Observation}
{The 13th International School/Symposium for Space Simulations (ISSS-13)}
{Los Angeles, USA, Sep. 13, 2018}

\TalkEntry
{Stochastic Shock Drift Acceleration for Electrons}
{8th East-Asia Workshop on Laboratory, Space, Astrophysical Plasmas}
{Daejeon, Korea, Aug. 1, 2018}

\TalkEntry
{Cosmic-Ray Acceleration via Astrophysical Coherent Radiation}
{20th International Symposium on Very High Energy Cosmic Ray Interactions (ISVHECRI)}
{Nagoya, Japan, May 24, 2018}

\TalkEntry
{Particle Acceleration in Relativistic Plasmas}
{Dawn of a New Era for Black Hole Jets in Active Galaxies}
{Sendai, Japan, Jan. 26, 2018}

\TalkEntry
{Nonthermal Electrons at Quasi-perpendicular Collisionless Shocks}
{7th East-Asia Workshop on Laboratory, Space, Astrophysical Plasmas}
{Weihai, China, Jul. 25, 2017}

\TalkEntry
{Coherent and Stochastic Acceleration in Quasi-perpendicular Collisionless Shocks}
{Workshop on Plasma Astrophysics from the Laboratory to the Non-thermal Universe}
{Oxford, UK, Jul. 4, 2017}

\TalkEntry
{Kinetic Simulations of Particle Acceleration and Transport around Collisionless Shocks}
{AOGS 13th Annual Meeting}
{Beijing, China, Aug. 1, 2016}

\TalkEntry
{Particle Acceleration and Transport at Collisionless Shocks}
{6th East-Asia Workshop on Laboratory, Space, Astrophysical Plasmas}
{Tsukuba, Japan, Jul. 11, 2016}

\TalkEntry
{Key Issues in Particle Acceleration Theory at Collisionless Shocks}
{18th International Congress on Plasma Physics}
{Kaohsiung, Taiwan, Jun. 29, 2016}

\TalkEntry
{Energetic Particle Hybrid Code and Its Application}
{11th International Conference on Numerical Modeling of Space Plasma Flows (ASTRONUM2016)}
{Monterey, USA, Jun. 9, 2016}

\TalkEntry
{Superluminal Electromagnetic Waves in Highly Magnetized Relativistic Shocks}
{5th East-Asia School and Workshop on Laboratory, Space, Astrophysical Plasmas}
{Pohang, Korea, Aug. 21, 2015}

\TalkEntry
{Quasi-neutral Two-fluid Plasma Simulation Model}
{10th International Conference on Numerical Modeling of Space Plasma Flows (ASTRONUM 2015)}
{Avignon, France, Jun. 10, 2015}

\TalkEntry
{Physics of Very High Mach Number Collisionless Shocks}
{The Many Facets of Supernova Remnants}
{Rikkyo University, Japan, Nov. 10, 2014}

\TalkEntry
{Relativistic Electromagnetic Two-fluid Simulations of Pulsar Wind Termination Shocks}
{The 6th East-Asian Numerical Astrophysics Meeting (EANAM6)}
{Suwon, Korea, Sep. 18, 2014}

\TalkEntry
{Robust Handling of Low Density Regions in Hybrid Simulations}
{9th International Conference on Numerical Modeling of Space Plasma Flows (ASTRONUM 2014)}
{Long Beach, USA, Jun. 25, 2014}

\TalkEntry
{Relativistic Pulsar Wind Termination Shocks Modified by Superluminal Electromagnetic Waves}
{8th International Conference on Numerical Modeling of Space Plasma Flows (ASTRONUM 2013)}
{Biarritz, France, Jul. 1, 2013}

\TalkEntry
{Structure of Relativistic Shock Modified by Nonlinear Superluminal Waves}
{Nonlinear Waves and Chaos Workshop 9}
{La Jolla, USA, Mar. 7, 2013}

\TalkEntry
{Self-consistent Drift-kinetic Numerical Ring-current Modeling : Five-dimensional Vlasov-Maxwell Approach}
{Inner Magnetosphere Coupling II (IMC II)}
{Los Angeles, USA, Mar. 20, 2012}

\TalkEntry
{Nonthermal Electron Acceleration and Injection in Collisionless Shocks}
{International Astrophysics Forum Alpbach (IAFA) 2011}
{Alpbach, Austria, Jun. 24, 2011}

\TalkEntry
{Kinetic and Self-consistent Numerical Modeling of the Terrestrial Inner Magnetosphere}
{6th International Conference on Numerical Modeling of Space Plasma Flows (ASTRONUM 2011)}
{Valencia, Spain, Jun. 17, 2011}

\TalkEntry
{Electron Acceleration and Injection by Whistler Waves in Collisionless Shocks}
{2010 International Space Plasma Symposium}
{Tinan, Taiwan, Jun. 28, 2010}

\TalkEntry
{Surfing and Drift Acceleration of Electrons at High Mach Number Quasi-perpendicular Shocks}
{Kinetic Modeling of Astrophysical Plasmas}
{Crakow, Poland, Oct. 6, 2008}

\TalkEntry
{Nonthermal Electron Acceleration in High Mach Number Collisionless Shocks}
{The 9th International Workshop on the Interrelationship between Plasma Experiments in Laboratory and Space (IPELS)}
{Palm Cove, Australia, Aug. 10, 2007}

\end{enumerate}

\newpage
\NewPart{Invited Talks (Domestic Conferences)}{}
\begin{enumerate}

\TalkEntry
{宇宙プラズマにおける運動論的不安定性}
{プラズマシミュレータシンポジウム}
{オンライン, 2023年9月28日}

\TalkEntry
{ミクロなプラズマ素過程に基づく衝撃波粒子加速機構}
{高エネルギー現象で探る宇宙の多様性Ⅰ}
{東京大学宇宙線研究所, 2021年10月19日}

\TalkEntry
{衝撃波電子加速におけるホイッスラー波の役割}
{第37回プラズマ・核融合学会年会シンポジウム}
{オンライン, 2020年12月1日}

\TalkEntry
{内部磁気圏におけるULF波動励起機構}
{実験室・宇宙プラズマ研究集会}
{東京大学本郷キャンパス, 2019年9月17日}

\TalkEntry
{宇宙空間衝撃波の遷移層}
{日本物理学会 春季年会}
{東京理科大学野田キャンパス, 2018年3月24日}

\TalkEntry
{内部磁気圏RCモデリングの新しい試み}
{太陽地球圏環境予測のためのモデル研究の展望}
{名古屋大学東山キャンパス, 2017年1月27日}

\TalkEntry
{MMS衛星で見る無衝突衝撃波と電子加速}
{高エネルギー宇宙物理学研究会}
{青山学院大学相模原キャンパス, 2016年12月2日}

\TalkEntry
{宇宙プラズマのハイブリッドシミュレーション}
{日本物理学会 2016秋季年会}
{金沢大学角間キャンパス, 2016年9月14日}

\TalkEntry
{Theory and Simulations of Particle Acceleration in Collisionless Shocks}
{高エネルギーガンマ線でみる極限宇宙2015}
{2016年1月14日}

\TalkEntry
{パルサー風衝撃波と電磁波の相互作用}
{高エネルギー宇宙物理学研究会}
{九州大学西新プラザ, 2014年11月25日}

\TalkEntry
{相対論的電磁変性衝撃波の構造と電磁エネルギー散逸}
{日本物理学会 2013春季年会}
{広島大学, 2013年3月27日}

\TalkEntry
{無衝突衝撃波の数値シミュレーションと粒子加速}
{宇宙流体力学のフロンティア}
{京都大学, 2009年11月16日}


\end{enumerate}

\end{document}
